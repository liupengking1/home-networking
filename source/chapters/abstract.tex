%% Abstract chapter
%% author Liu Peng
In recent years, the rapid development of electronics and computer science has
enabled home networking devices to become more affordable and more powerful.
Several widely used multimedia-streaming solutions have become available in the
market. However, these standards are not compatible with each other. Moreover,
even devices using the same standard are not always compatible with each other,
since the implementation approaches may vary from device to device.

This thesis compares the modern solutions for multimedia home networking (MHN),
including AirPlay, Digital Living Network Alliance (DLNA), Miracast, and
Chromecast standards. By analyzing the features and capabilities of these
streaming technologies, this thesis proposes a universal solution for MHN to
support multiple protocols and bridge different platforms.

Based on this thesis, Tuxera team tested different multimedia solutions and
implemented a mobile application for home networking on the Android operating
system. Over 16 months after the launch of this application on the Google Play
Store, the application received over one million downloads from 225 countries.
Over 10000 users rated the application, and near half of the users gave it 5
stars. This solution has proved to be competitive and successful.

In the future, the company plans to keep on the development of this application,
including the support for more platforms, independent streaming SDK, and
performance optimization. In addition, this thesis discusses the future trends
of home networking, which is more connected and cloud based.
