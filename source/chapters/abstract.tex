%% Abstract chapter
%% author Liu Peng
This thesis compares the modern solutions for multimedia home networking(MHN),
especially the popular ones that follow the Digital Living Network Alliance standard. By conducting research on the features and implementations of these existing solutions, Tuxera developed a suitable mobile solution for MHN
 which takes advantage of AirPlay, DIAL, and DLNA on the Android
 platform.

The thesis starts with an overview of the popular streaming technologies, including AirPlay,
DLNA, Miracast, and Chromecast. By analyzing the features and capabilities of
these streaming technologies, a universal solution was proposed for MHN in the hope of  supporting multiple protocols and bridging
 different platforms.

Different multimedia solutions were tested and a mobile application
for home networking was implemented for Android operating system. The corresponding system architectures, features, and analysis methodologies are also analyzed to present the competitiveness of this application.

In terms of practical contribution, an online channel proxy was integrated to  the "Streambels" application to fulfill the target of streaming online channels such as YouTube. By implementing this online channel proxy, home
 networking and Internet resources are effectively bridged together. 

Based on this thesis study, an Android application for Tuxera Inc. was released. Over the 16 months after the launch of this application on the Google Play Store, the application received over one million downloads from 225 countries. Over 10000 users rated the application, near half of the users gave it a 5 star. This solution turns out to be competitive and successful.

Further development of this application, including the support for more platforms, independent streaming SDK, and performance optimization, was planned. Future of multimedia networking was discussed to predict the trend of home networking, which is more connected and cloud based.