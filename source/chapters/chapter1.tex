%% Introduction chapter
%% autor Liu Peng

\subsection{Home networking}
People's lives are being digitalized. Home multimedia devices nowadays such as digital TVs,
smart phones, digital cameras, tablets, PCs, laptops and NAS (Network Attached
Storage) are all being equipped with ever greater processing power and mass storage, wielding the power  to
 record our daily lives and handle the multimedia information. The digitalized world has also seen a rapid growing of network deployment. According to a research
\cite{stateofHN} done in 2011, in this industrialized world, networking
is being rapidly adopted at homes; for example, in the U.S. in 2009,
approximately 63\% of homes had gained access to a broadband connection. Over 50\% had even installed their own "home network", which is defined as multiple computers or devices sharing a broadband
 connection via either a wired or wireless connection within the home. In a typical
home scenario, most of these devices are connected to a local network such as a
Wi-Fi hot spot, in order to allow music, pictures, videos and other content to
be ported across different devices.

\subsection{Problem description}
While the adoption of home networks has steadily increased since the late 1990s
and early 2000s, the home networks have indeed encountered problems and limitations
\cite{stateofHN}. For example, the usability of home-networking technologies has become
a key impediment to the adoption of new applications in the home, since the home-networking technology was originally developed for
 research labs and enterprise networks and does not account for the unique
 characteristics of home usage, such as the lack of professional administrators, deep
heterogeneity, and expectations of privacy.\\
\\
Among all the challenges of home-networking, connecting all media devices and making them work together is becoming increasingly interesting because of the rapid growth of consumer electronics markets. Although there are
 several widely used multimedia-streaming solutions in the market, the 
standards employed are not compatible with each other. Furthermore, even devices using
the same standard are not always compatible with each other, due to the fact
that the implementation approaches could vary from device to device. These
incompatibilities are causing great inconvenience to the end users.\\
\\
DLNA, AirPlay, Chromecast and Miracast are the four major multimedia home network
digital living solutions. AirPlay is only used between Apple products; it
provides various features, including iTunes play for music, AirPlay for video
and photos and screen mirroring. Miracast (previously called Wi-Fi display) was
proposed by Wi-Fi Alliance and has been receiving great popularity over the recent years.
Since its release version 4.2.2, Android has officially added the support for Miracast. In comparison, DLNA has been the most widely deployed solution so far, with 2.2 billion installations worldwide.
It was proposed by several industrial leading electronic manufacturers and network operators 
including AT$\&$T, Broadcom, Cisco, Google, Huawei, Intel, LG Electronics, Microsoft, Nokia, 
Panasonic, Samsung, Sony and Verizon.\\
\\
As a result of different technical designs and  human aspects, these standards proposed by
different device manufactures naturally experience serious compatibility issues. It usually happens that
end users can have several multimedia-devices, with each one using a
distinctive and unique  protocol, making it challenging or even impossible
sometimes to share media between those devices.\\
\\
These compatibility issues have really obliged us to make a study on those multimedia-streaming
 technologies and to develop a more easy-to-use multimedia home networking solution
 with more  advanced technologies.

\subsection{Structure of the thesis}
The overview of popular home networking standards and solutions is described in
Chapter \ref{chapter2}. After a short comparison of these solutions, in Chapter \ref{chapter3} we
describe a more universal solution for multimedia home networking and its
implementation. Chapter \ref{chapter4} presents some statistics from the Google Store
during the past few months. Besides, a study based on the user feedback is also presented. In Chapter \ref{chapter5}, the thesis is concluded by giving a discussion on the further development and prospect of home networking.