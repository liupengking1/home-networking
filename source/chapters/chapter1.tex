%% Introduction chapter
%% autor Liu Peng

Before we start problem description, we introduce concept of home networking.
Understanding home networking and its challenges can help with the problem
description.

\subsection{Home networking}
People's life is digitalized, home multimedia devices like digital TV, smart phone, 
digital camera, tablet, PC, laptop and NAS (Network Attached Storage) are all equipped
 with great processing power and mass storage to handle multimedia information which 
 records our daily life. On the other hand, the deployment of networking is
 growing fast. According to a research \cite{stateofHN} done in 2011, in the
 Western industrialized world, networking is rapidly being adopted in the home;
 for example, in the U.S. in 2009, approximately 63\% of homes had a broadband
 connection, and over 50\% had a "home network", which is defined as multiple
 computers sharing a broadband connection via either a wired or wireless network within
 the home. In a typical home scenario, most of these devices are
 connected to a local network like Wi-Fi hot spot, so that allow music,
 pictures, videos and other content to be ported across different devices.

\subsection{Problem description}
While adoption of home networks has steadily increased since the late 1990s and
early 2000s, this growth also reflects deep problems and limitations
\cite{stateofHN}, for example, the usability of home-networking technologies is
a key impediment to adoption of new applications in the home. The network
usability problems run deep because the technology was originally developed for
research labs and enterprise networks and does not account for the unique
characteristics of the home: lack of professional administrators, deep
heterogeneity, and expectations of privacy.

Among all the challenges, the problem of connecting all media devices in home
networking and make them work together is getting more and more interesting
because of the rapid growth of consuming electronic market. Although there are
several widely used multimedia-streaming solutions in the market, these
standards are not compatible with each other. What's more, and due to different
implementation approach, even devices using same standard are not always
compatible with each other. This caused great inconvenience to end-users.

DLNA, Airplay, Chromecast and Miracast are four major multimedia home network
digital living solutions. Airplay is only used between Apple products; it
provides various features, including tunes play for music, airplay for video
and photos and screen mirroring. Miracast (previously called Wi-Fi display) is
proposed by Wi-Fi Alliance, and in recent years it gets more and more popular,
from version 4.2.2, Android has officially added support of Miracast. On the
other hand, DLNA is most widely deployed solution so far, with 2.2 billion installations.
It was proposed by several industrial leading electronic manufacturers and network operators 
like AT$\&$T, Broadcom, Cisco, Google, Huawei, Intel, LG Electronics, Microsoft, Nokia, 
Panasonic, Samsung, Sony and Verizon.

As a result of the collision between technical design choices and fundamental
aspects of the human condition. These standards which are proposed by different
device manufactures are not compatible with each other. It usually happens that
end-users have several multimedia-devices that are using different protocols;
sharing media between those devices become a headache.

So it is quite interesting to study and compare those multimedia-streaming technologies and 
develop a more easy-to-use multimedia home networking solution for modern advanced home 
networking.

\subsection{Document overview}
The overall popular home networking standards and solutions will be described in
chapter 2. After a short comparison of these solutions, in chapter 3, we
describe a more universal solution for multimedia home networking and its
implementation. Chapter 4 give some statistics from the Google Store
during the past few months, and we have a user study according to the feedback
we got. In Chapter 5, a discussion is given to the further development and the
future of home networking.
