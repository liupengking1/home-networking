%% Introduction chapter
%% autor Liu Peng

\subsection{Home networking}
People's lives are being digitalized. This digitalization can be seen in the
increasing number of home multimedia devices, such as digital TVs, smart
phones, digital cameras, tablets, PCs, laptops and NAS (Network Attached
Storage), which are all equipped with ever greater processing power and mass
storage, wielding the power to record our daily lives and handle multimedia
information. The digitalized world has also seen rapid growth in network
deployment.  In this digitalized world, networking is being rapidly adopted at
homes. For example, in the U.S. in 2009, approximately 63\% of the homes had
already gained access to a broadband connection \cite{stateofHN}. Over 50\% of
these households had even installed their own "home network", which is defined as
multiple computers or devices sharing a broadband connection via either a wired
or wireless connection within the home \cite{standards-perspective}. In a typical home
scenario, most of these devices are connected to a local network, such as a
Wi-Fi hot spot, in order to allow music, pictures, videos and other content to
be ported across different devices.

\subsection{Motivation and Aims}
While the adoption of home networks has steadily increased since the late 1990s
and early 2000s, home networks have indeed encountered problems and limitations
\cite{stateofHN}. For example, the usability of home-networking technologies has
become a key impediment to the adoption of new applications in the home, since the
home-networking technology was originally developed for research labs and
enterprise networks and does not account for the unique characteristics of home
usage, such as the lack of professional administrators, deep heterogeneity, and
expectations of privacy. Among all the challenges of home-networking,
connecting all media devices and making them work together is becoming
increasingly interesting because of the rapid growth of consumer electronics
markets. Although several widely used multimedia-streaming solutions have
become available in the market, the standards employed are not compatible with
each other. Moreover, even devices using the same standard are not always
compatible with each other, since the implementation approaches may vary from
device to device. These incompatibilities are causing great inconvenience to
the end users.\\
\\
Currently, four major multimedia home network digital living solutions  are
deployed: AirPlay, Miracast, DLNA, and Chromecast. AirPlay is only used
between Apple products; it provides various features, including iTunes for
playing music as well as AirPlay for video, photos and screen mirroring.
Miracast (previously called Wi-Fi Display) was proposed by the Wi-Fi Alliance
and has received great popularity over recent years. Since its release version
4.2.2, Miracast has officially supported the Android operating system. Of the
four standards, DLNA has become the most widely deployed solution, with 2.2
billion installations worldwide. DLNA was proposed by several industrial
leading electronic manufacturers and network operators, including AT$\&$T,
Broadcom, Cisco, Google, Huawei, Intel, LG Electronics, Microsoft, Nokia,
Panasonic, Samsung, Sony and Verizon.\\
\\
As a result of their different technical designs, these standards proposed by
individual device manufacturers naturally experience serious compatibility
issues. Thus, end users can have several multimedia devices, with each one
using a distinctive, unique protocol, making it challenging or even impossible
sometimes to share media between those devices. These compatibility issues have
motivated the need to determine the technological features common to the four
multimedia-streaming standards and to develop a more easy-to-use multimedia
home networking solution based on more advanced technologies.

\subsection{Structure of the thesis}
The remainder of this thesis is divided into four chapters. Chapter
\ref{chapter2} provides an overview of popular home networking standards
currently in use. After a short comparison of these solutions, Chapter
\ref{chapter3} develops and proposes a more universal solution for multimedia
home networking and its implementation.
Chapter \ref{chapter4} evaluates the streaming performance of our solution and
presents some recent statistics from the Google Store to demonstrate the
compatibility of the proposed solution. This chapter also presents a study
based on the user feedback in order to further improve this solution. In
Chapter \ref{chapter5}, the thesis is concluded by  discussing potential 
further developments and prospect of home networking.