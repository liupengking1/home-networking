%% Developing a solution for multimedia home networking chapter
%% author Liu Peng

We developed an Android application that has integrated with a simple media server, Airplay/DLNA device discovery, and Airplay/DLNA streaming control point.
As we know that Airplay and DLNA works differently and have different features, but according to the previous study, we could combine some use cases of both protocols. We are developing an Android application that can handle most multimedia devices in a typical home networking. Feature list:
\begin{itemize}
\item[--]Firstly the app is a multimedia player, it can play music, photos and videos on SD card locally on Android phone
\item[--]It can stream local content to Apple TV, Airport express and Airplay-enabled speakers.
\item[--]It can stream local content to DLNA media renderers, which has a huge device base.
\item[--]It can stream local content to Chromecast devices.
\item[--]It can browse content from the DLNA media servers, a typical source is a Network Attached Storage (NAS). And play the media locally on the Android device.
\item[--]It can browse content from the DLNA media servers and stream it to DLNA media renderers.
\item[--]It can browse content from DLNA media servers and stream it to Airplay enabled devices using a different protocol.
\item[--]It can proxy online channels' content to DLNA and Airplay enabled devices. (Currently YouTube and Facebook videos are supported, but integration to Spotify is still in progress).
\end{itemize}