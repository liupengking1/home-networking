%% Reference chapter
%% author Liu Peng
%% example reference, later used bib file to replace this

%% Alla pilkun j�lkeen on pakotettu oikea v�li \<v�lily�nti>-merkeill�.
\bibitem{Kauranen} Kauranen,\ I., Mustakallio,\ M. ja Palmgren,\ V.
  \textit{Tutkimusraportin kirjoittamisen opas 
    .}  Espoo, Teknillinen korkeakoulu, 2006.

\bibitem{Itkonen} Itkonen,\ M. \textit{Typografian .} 3.\
  painos.  Helsinki, RPS-yht, 2007.

\bibitem{Koblitz} Koblitz,\ N. \textit{A Course in Number Theory and
    Cryptography. Graduate Texts in Mathematics 114.}  2.\ painos. New
  York, Springer, 1994.

%% Kun on useampi nimikirjain, jokaisen nimikirjaimen v�liin
%% kuuluu v�lily�nti. Oikea v�lin m��r� on saatu \<v�lily�nnill�>
\bibitem{bcs} Bardeen,\ J., Cooper,\ L.\ N. ja Schrieffer,\ J.\ R.
  Theory of Superconductivity. \textit{Physical Review,} 1957, vol.\
  108, nro~5, s.\ 1175--1204.

\bibitem{Deschamps} Deschamps,\ G.\ A. Electromagnetics and
  Differential Forms. \textit{Proceedings of the IEEE,} 1981, vol.\
  69, nro~6, s.\ 676--696.

%% Alla esimerkki englanninkielisen tavuttamisen pakottamisesta.
%% Oletusarvoisesti k�ytet��n suomalaista tavutusta, mutta viitteiss�
%% esiintyy usein muunkielisi� lauseita, jotka tulevat siten tavutetuksi
%% suomen kielen s��nt�jen mukaan. T�m�n voi korjata \foreignlanguage-
%% komennolla, jonka ensimm�inen parametri on vieraan kielen nimi ja toinen 
%% on vieraalla kielell� tavutettava teksti. 
\bibitem{Sihvola} Sihvola,\ A.\ et al.
  

%% Alla on suomalainen yhdistelm�sukunimi. Sen nimien v�liss� 
%% k�ytet��n yhdysmerkki� l. tavuviivaa, kirjoitetaan -.
\bibitem{Lindblom} Lindblom-,\ S. ja Wager,\ M.  Tieteellisten
   ohjaaminen. Teoksessa: Lindblom-Yl�nne,\ S. ja
  Nevgi,\ A. (toim.) \textit{Yliopisto- ja korkeakouluopettajan
    .}  Helsinki, WSOY, 2004, s.\ 314--325.
 
\bibitem{Miinusmaa} Miinusmaa,\ H. Neliskulmaisen rei�n poraamisesta
  kolmikulmaisella poralla. Diplomity�, Teknillinen korkeakoulu,
  konetekniikan osasto, Espoo, 1977.

%% T�ss� taas pakotettu englanninkielinen tavutus. 
%% Pedanttinen kirjoittaja pakottaa tietysti jokaiseen englanninkieliseen
%% lauseeseen englannin tavutuksen, mutta t�ss� esityksess� ei n�in ole
%% tehty selvyyden ja l�hdekoodin luettavuuden takia. 
\bibitem{Loh} Loh,\ N.\ C. High-Resolution Micromachined
  Interferometric Accelerometer. Master's Thesis, Massachusetts
  Institute of Technology, Cambridge,

\bibitem{Lonnqvist} ,\ A.
  

\bibitem{sfs} SFS 5342. Kirjallisuusviitteiden laatiminen. 2.\ painos.
  Helsinki, Suomen standardisoimisliitto, 2004. 20~s.

\bibitem{haastattelu} Palmgren,\ V. Suunnittelija. Teknillinen
  korkeakoulu, kirjasto. Otaniementie 9, 02150 Espoo. Haastattelu
  15.1.2007.

\bibitem{Ribeiro} Ribeiro,\ C.\ B., Ollila,\ E. ja Koivunen,\ V.
  

\bibitem{Stieber} Stieber,\ T. GnuPG Hacks. \textit{Linux Journal,}
  verkkolehti, 2006, maaliskuu, nro~143. Viitattu 19.1.2007. Lehti
  ilmestyy  painettuna. Saatavissa:
  \url{http://www.linuxjournal.com/article/8732.}

\bibitem{kone} Pohjois-Koivisto,\ T. Voiko kone tulevaisuudessa arvata
  tahtosi?  \textit{Apropos,} verkkolehti, helmikuu, nro~1, 2005.
  Viitattu 19.1.2007.  Saatavissa:
  \url{http://www.apropos.fi/1-2005/prima.php.}

\bibitem{Adida} Adida,\ B.  Advances in Cryptographic Voting Systems.
  Verkkodokumentti. Ph.D.\ Thesis, Massachusetts Institute of
  Technology, Cambridge, 

\bibitem{viittaaminen} ,\ P. WWW- viittaaminen
  . Verkkodokumentti.  26.11.2001.
  Viitattu 19.1.2007. Saatavissa:
  \url{http://www.cs.uku.fi/~kilpelai/wwwlahteet.html.}