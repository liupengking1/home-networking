%% Discussion chapter
%% author Liu Peng
This chapter reviews the work that is done during this thesis project and
summarizes the conclusion of this thesis. Section \ref{5_1} draws the
conclusion of this thesis project. Section \ref{5_2} describes the further
possible development that could be conducted to improve our current
home-networking solution. Finally, Section \ref{5_3} discusses the future trend
of multimedia home networking.
\subsection{Conclusion\label{5_1}}
Since its launch in November 2013, this application has received over one
million downloads and a daily active user total of over 15000. People from 99\%
percent of all the countries worldwide have downloaded and tried this solution.
There is a process going on to improve Streambels and all these
statistics have proven that this solution is competitive for multimedia home
networking.

The launch of Streambels has been successful, but the problem of multimedia
 home networking is still not completely solved. As more manufacturers are
 involved in the development of home networking, new standards are being pushed
 on the market constantly. It may not be possible to develop a single
 system that support all future protocols. To improve this solution and solve
 the problems of home networking, a better understanding and outlook of the
 future of home networking is necessary. And new challenges must be identified
 and accepted.

\subsection{Further development\label{5_2}}
\textbf{Support for more platforms}

The development and implementation of our solution on Android has been completed
and the outcome has proved to be satisfactory. The next step in development is
to extend the solution to other platforms, including Apple iOS and Microsoft
Windows, which are widely recognized as the most popular mobile operating
systems. It is essential to move on to iOS first since it has a larger user
base. The work on the iOS version implementation has already started with the
release already released on the market. It will be continuously improved to
the same quality as the Android version.

\textbf{Developing an Software Development Kit}

One of the limitations of this solution is that it can only work with limited
content sources. There are thousands of content providers in the market, and it
would be impossible to integrate the support of all content sources by the
Tuxera development team alone. An ideal solution for this would be to develop a
Software Development Kit (SDK), so that the content providers can directly
integrate this solution into their software clients, enabling access to the
devices in home networking. Since the release of Streambels, there have been
actively developing of this SDK. In the foreseeable future, this solution will
likely be used by more and more Internet content providers.

\textbf{Integration of receiver functionality}

Another limitation of this current solution is the content sharing between
phones and tablets, and between the Android and iOS mobile platforms.
In further development, the receiver functionalities is also planned to be
integrated into this solution. By developing the in-house protocol using
existing architecture, the compatibility can be guaranteed across different
mobile platforms.

\textbf{Support for multi-session}

In this current solution, only one receiver device can be selected at the one
time. However, there exists a use case whereby a user may want to stream the
same item to a different receiver at the same time. Another use case exists
whereby different media items are required to stream to different receivers;
thus multi-session support becomes a valid demand. Further improvements
for this should certainly take this demand into consideration, and integrate the
support for multi-session streaming.

\textbf{Optimization for codec}

One significant problem that remains unsolved in multimedia home networking is
that the codec varies from device to device. Having aggregated all content
sources, the application should also convert the format of the aggregated
content to a format that is supported by the receiver. In further development,
better transcoding support should be integrated into this solution.

\textbf{In-kernel media server}

One of the key impediments for multimedia home networking is the streaming
performance. It is challenging to guarantee streaming quality when streaming
contents at a high bit rate. Since Tuxera has competence in developing file
systems in kernel space with many leading Android phone manufacturers using its
file system kernel module, it is possible for Tuxera to develop and integrate
the media server functionalities into the Linux kernel space. Thus, the
application can utilize these functions exposed from the kernel in the user
space. Moreover, similar in-kernel web servers such as the TUX web server
\cite{tux_webserver} have already been implemented by different developers.
This also proves a in-kernel media server is possible to implement.
\subsection{Future of multimedia home networking\label{5_3}}
\textbf{Device discovery}

According to all the previous study on the different standards in the market,
there are mainly two mechanisms of service discovery introduced: Simple Service
Discovery Protocol (SSDP) as a part of UPnP and multicast DNS (mDNS) as part of
Zeroconf. Research \cite{zeroconf_vs_upnp} has been conducted previously to
assess which is the better solution. Generally speaking, mDNS provides an
easier way to implement a new device, since the technology is simpler. While
implementing a new device in UPnP would be much challenging, since a new UPnP
forum will be started to handle the implementation for each new device type.

It seems mDNS will finally overtake SSDP due to its flexibility. One notable
sign of this development is that in the latest DLNA road map, mDNS will also be
included in the service discovery protocol set, with the mandatory requirement
of backward compatibility. However there is still a long way to go since UPnP
has a strong alliance, including hundreds of companies.

\textbf{Information exchange}

As mentioned in Chapter \ref{chapter2}, for DLNA and AirPlay, the control
messages are sent through SOAP (Simple Object Access Protocol), which is based on
Hypertext Transfer Protocol (HTTP) and its Extensible Markup Language (XML).
However, much research \cite{restful_webservice} shows that Representational
State Transfer (REST) is the better solution due to its light weight and
flexibility. Accordingly, performance benchmark \cite{performance_restful_saop}
has been done to compare the RESTful service to SOAP service. The result of this
research shows that the performance of RESTful is obviously higher compared to
SOAP. Therefore, future development of a home networking solution might opt for
RESTful rather than SOAP, in order to provide higher flexibility and lower
overhead.

\textbf{Streaming protocol}

According to the study conducted in Chapter \ref{chapter4}, HTTP streaming
performs better in limited bandwidth conditions. While performs better in high
packet loss situation. In the case of home networking where the network
condition is usually sufficient and packet loss is not a common phenomenon; thus
HTTP would be a better solution.

However, there still remains one significant problem with HTTP streaming. In the
case of streaming the same content to different receivers at the same time,
synchronization would represent a key issue for HTTP, since it is not possible
to decide the buffer size of the receiver. One more situation that HTTP cannot
solve is the transmission of the data in real time, as required by a screen
mirroring application. In these cases, RTSP would be a better solution due to
the fact that UDP is used and there is no buffering or retransmission of the
same content.

In general, both HTTP and RTSP will be used in future home networking. For
the streaming of recorded media, HTTP streaming has an advantages over RTSP
due to the buffer system. On the other hand, for screen mirroring and other
usages where the content needs to be synchronized precisely, RTP would be a
better solution.

\textbf{Internet of Things}

As more devices at home acquires the capability to handle network based
applications, home networks could see a dramatic growth in device types. The
multimedia home networking will move towards the Internet of Things, and all
the data could be aggregated and accessed anywhere in the world. Consquently,
the solution for future home networking could be an Internet-connected
world.
