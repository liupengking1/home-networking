%% Discussion chapter
%% author Liu Peng
This chapter reviews the work that is done during this thesis project and
summarizes the conclusion of this thesis. Section \ref{5_1} draws the
conclusion of this thesis project. Section \ref{5_2} describes the further
possible development that could be conducted to improve our current home-networking solution. Finally, Section \ref{5_3} discusses the future trend of multimedia home networking.
\subsection{Conclusion\label{5_1}}
Since its launch in November 2013, our application has received over 1 million
downloads and a daily active user number of over 15000. People from 99\% percent
of all the countries have downloaded and tried our solution. We kept improving
Streambels and all these statistics have proven that our solution is competitive for multimedia home networking.\\
\\
The launch of Streambels has been successful, but the problem of multimedia
 home networking is still not completely solved. As more manufacturers are involved in the development of home networking, new standards are being pushed to the market constantly. It is literally not possible to develop a single system that support all future protocols . To improve our solution and solve the problems of home networking, a better understanding and outlook of the future of home networking is necessary. And new challenges must be identified and accepted.

\subsection{Further development\label{5_2}}
\subsubsection{Support for more platforms}
The development and implementation of our solution on Android has been completed and the outcome proves to be satisfactory. Our next step of developing is to extend the solution to other platforms, including Apple iOS and Microsoft Windows, which are widely recognized as the most popular mobile operating systems. We would like to expand our product on the iOS platform first since it has a larger user base. The work on the iOS version implementation has already started, and the release is already published in the market. We are keeping on improving the iOS version product currently, hoping that it will soon match the  quality level of our Android version.
\subsubsection{Developing an Software Development Kit}
One of the limitations of our solution is that it can only work with limited content sources. There are thousands of content providers in the market and it could hardly be possible to integrate the support of all content sources by own development team alone. An ideal solution would be to develop a Software Development Kit (SDK), so that the content providers can directly integrate our solution in their software clients, enabling the access to the devices in home networking. Since the release of our Streambels, we have been actively developing this SDK. In the foreseeable future our solution will likely be used by more and more Internet content provides.
\subsubsection{Integration of receiver functionality}
Another limitation of our current solution is the content sharing between phones and tablets, and the content sharing between Android and iOS mobile platforms. In the future development, the receiver functionalities will also be integrated into our solution. By developing our own protocol using existing architecture, the compatibility can be guaranteed across different mobile platforms.
\subsubsection{Support for multi-session}
In our current solution, only one receiver device can be selected at the one time. However, there is a use case that user may want to stream the same item to different receiver at the same time. There is also a use case that different media items need to be streamed to different receivers. In this sense, the multi-session support is in real demand. Further improvements should certainly take this demand into consideration, and integrate the support for multi-session streaming.
\subsubsection{Optimization for codec}
One big problem that is not solved for multimedia home networking is that the codec is quite different from device to device. Having aggregated all content sources, the application should also convert the format of the aggregated content to a format that is supported by the receiver. In further development, better transcoding support should be built in our solution.
\subsubsection{In-kernel media server}
One of the key pediments for multimedia home networking is the streaming performance. It is hard to guarantee streaming quality when streaming high bit rate contents. Since Tuxera have the experts in developing file systems in kernel space and has many leading Android phone manufacturers using our file system kernel module, it is possible for our company to integrate the media server functionalities into the Linux kernel space. In this way, the application can use these functions exposed from the kernel in the user space.
Moreover, similar in-kernel web servers such as the TUX web server \cite{tux_webserver}, have already been implemented by different developers, proving that a in-kernel media server is possible to implement.
\subsection{Future of multimedia home networking\label{5_3}}
\subsubsection{Device discovery}
According to all the previous study on the different standards in the market, there are mainly two mechanisms of service discovery introduced: Simple Service Discovery Protocol(SSDP) as a part of UPnP and multicast DNS(mDNS) as part of Zeroconf. Research \cite{zeroconf_vs_upnp} has been done previously to discuss which one is the better solution. Generally speaking, mDNS provides an easier way to implement a new device, since the technology is simpler. While implementing a new device in UPnP would be much harder, since for each new device type, a new UPnP forum will be started to handle the implementation.\\
\\
It seems mDNS will finally over take SSDP due to its flexibility. One big sign of this take over is that in the latest DLNA road map, mDNS will also be enlisted in the service discovery protocol set, with the mandatory requirement of  backward compatibility. However there will be still a long way to go since UPnP has a strong alliance, including hundreds of companies.
\subsubsection{Information exchange}
As studied in Chapter \ref{chapter2}, for DLNA and AirPlay the control messages are sent through SOAP (Simple Object Access Protocol), which is based on Hypertext Transfer Protocol (HTTP) and its Extensible Markup Language (XML). However many research \cite{restful_webservice} shows that Representational State Transfer (REST) is the better solution due to its light weight and flexibility. Performance benchmark \cite{performance_restful_saop} has been done to compare the RESTful service to SOAP service and the result shows that the performance of RESTful is obviously better. Therefore future development of home networking solution might move to use RESTful rather than SOAP,  in order to provide higher flexibility and lower overhead.
\subsubsection{Streaming protocol}
According to the study conducted in Chapter \ref{chapter4}, HTTP streaming has better performance under limited bandwidth. While RTSP has a better performance in high packet loss situation. In the case of home networking where the network condition is usually good enough and packet loss is usually minute, HTTP would be a better solution. \\
\\
However, there is still one big problem of HTTP streaming. In the case of streaming same content to different receivers at the same time, synchronization would be a really big issue for HTTP, since it is hard to tell the buffer size of a receiver. One more situation that HTTP can not solve is to transmit the data in real time,  as required by a screen mirroring application. In these cases, RTSP would be a better solution due to the fact that UDP is used and there is no buffering or retransmission of the same content.\\
\\
To conclude, both HTTP and RTSP will be used in the future home networking. For streaming recorded media, HTTP streaming holds the advantages due to the buffer system. On the other hand, for screen mirroring and other usages where the content needs to be synchronized precisely, RTP would be a better solution.
\subsubsection{Internet of things}
As more devices acquires the capability to handle network based applications, home networks will see a dramatic growth in device types. The multimedia home networking will move towards the Internet of Things, and all the data would be aggregated and can be accessed anywhere in the world. The solution for the future home networking would be an Internet connected world.
