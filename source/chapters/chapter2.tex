%% Background chapter 
%% author Liu Peng 
In recent years, the rapid development of electronics and computer science has
enabled home networking devices to become more affordable and more powerful. It
is currently common that a person may own several multimedia devices that
can be connected to the network.

Early research \cite{link_layer_old} \cite{end_user} \cite{link_layer}
conducted on home networking mainly aimed to find out how to build home
networking infrastructure. The subjects of the research, including cable
connection, wireless connection, and optical connection, concern more about
the physical layer of the home network.  So far, it has turned out that the
IEEE 802.11 protocol stack, among all others, is the most successful and 
widely deployed home networking infrastructure.

Currently, a typical scenario of home networking is that an IEEE 802.11
supportive wireless router connecting to an Ethernet cable, optical cable or
Asymmetric Digital Subscriber Line (ADSL) from the network operator creates
a local network and other user devices simply join this network. The wireless
Access Point (AP) employs the 802.11 b/g/n/ac protocol, utilizing the 2.4 GHz
or 5 GHz frequency channels and providing a 100+ Mbps  network connection,
whose bandwidth is sufficient for transmitting the popular High Definition
(1080p) videos.

In terms of network and application layer technologies, different device 
manufactures tend to choose their preferred multimedia-sharing protocols from
the pool of protocols that have been developed for a long time.

Since late 1990s, Universal Plug and Play (UPnP) protocol had been developed for
home networking usage \cite{upnp}. At that time, XML was popular and widely
used by different network applications. Under such background, UPnP was designed
to fully make use of XML. UPnP is independent of media types and devices, and it
runs on the TCP/IP stack, thus it can be easily applied to modern network
infrastructures.

In June 2003, Sony and several leading consumer electronic manufacturers
established the Digital Living Network Alliance (DLNA), a nonprofit 
collaborative trade association\footnote{\url{http://www.dlna.org/dlna-for-industry/our-organization}}. The
DLNA standard is based on the widely used UPnP protocol, but it added some
restrictions on media formats and some compatibility requirements. A device
hardware and software can be certified by DLNA organizations to prove that it
can work with other devices that also passed this certification.

In 2010, Apple quit DLNA and developed its own multimedia home networking 
solution, known as AirPlay\footnote{\url{https://www.apple.com/airplay/}}. By
adding screen mirroring, authentication and Remote Audio Output Protocol (RAOP)
music streaming, Apple tried to forge a more advanced home network sharing
system, aiming to provide a unique user experience among Apple products.
Apple's solution indeed attracted people's interest, and the user experience
proved much better than that of other similar products in the market. With its
improvement over the years,  Apple's solution has now been acknowledged as one
of the most popular streaming solutions.

Two years later, Wi-Fi alliance released its Miracast
technology\footnote{\url{http://www.wi-fi.org/discover-wi-fi/wi-fi-certified-miracast}},
and participated in pushing a new standard in wireless home networking. The
Miracast uses the Wi-Fi direct technology \cite{miracast_consumer} and it does
not require a wireless local network. Instead, a peer-to-peer connection is
created between the sharing and receiving devices. After its release, some
major software and hardware companies soon accepted this new standard. Google,
for example integrated Miracast support into its Android operating system, and
provided a screen-mirroring feature to other Miracast
receivers\footnote{\url{https://support.google.com/nexus/answer/2865484?hl=en}}.

The competition in home networking rages on over the years. In 2013, Google
released a 35-dollar
Dongle\footnote{\url{http://www.google.com/chrome/devices/chromecast/}}, using
its Chromecast protocol, which makes it possible to watch YouTube and Netflix
video directly on TV with such a dongle device. Laptop and mobile devices with
official YouTube App or Chrome browser can control the Dongle through the home
local network. In this solution, the home networking is pushed to the cloud,
since YouTube and Netflix content are directly downloaded from the Internet
whereas the mobile device just acts as a controller for choosing the
contents \cite{dial}.

At the same time, in September 2013, Spotify, a startup music service 
company also took part in making its own home networking solution, called 
Spotify Connect \cite{spotifyconnect}. Spotify Connect provides an interface for
users at home to access its huge music database, and directly browse and stream
using its mobile application. Home networking has again been pushed towards the
cloud and Internet services in Spotify Connect.

Since many companies would like to develop their own devices and even their 
own protocols, the market becomes disordered. Devices from different companies
are not compatible with each other, and users have to buy a different device in
order to access different services like Netflix and Spotify, which are provided
by different companies. This has created a significant demand on a solution that
can connect those devices at home and make them work together in a user friendly manner.

In response of this market need, the Streambels project has been initiated,
aiming to fill the gap among different protocols and connect these different
types of devices in the home networking environment.
